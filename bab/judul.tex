\chapter{Judul Skripsi}
 
Judul dirumuskan dalam satu kalimat yang jelas, ringkas dan komunikatif. Jumlah kata dalam
judul berkisar antara 10 sampai 20 kata. Judul harus mencerminkan ruang lingkup penelitian,
subjek penelitian, dan variabel yang diteliti. Judul diketik menggunakan huruf kapital dan tidak
boleh menggunakan singkatan.

\section{Judul cover depan}
Halaman judul cover depan berisi judul, logo STKIP Surya, identitas penulis, identitas
sekolah tinggi, tempat, dan tahun penulisan. Halaman Sampul terbuat dari karton tebal (hard
cover) dilapisi kertas linen, warna sesuai dengan aturan program studi masing-masing.
Pengetikan pada halaman judul diketik simetris di bagian tengah (center). Logo STKIP
Surya dengan ukuran panjang 6 cm dan lebar 3,5 cm. Ukuran huruf tiap bagian tercantum
pada format cover depan. Pengecualian untuk nama program studi. Program Studi
Pendidikan Fisika dan Matematika dengan ukuran huruf 14. Sedangkan program studi
Teknologi Informasi dan Komunikasi (TIK) dengan ukuran huruf 12. Format dan contoh
cover depan dapat dilihat pada halaman berikutnya.
\newpage
\halamanJudulLuar
\newpage
%\includepdf[pages=-]{judulDepan}
\section{Judul cover dalam}
Halaman judul cover dalam berisi judul, pernyataan mengenai maksud penulisan skripsi,
logo STKIP Surya, identitas penulis, identitas sekolah tinggi, tempat, dan tahun penulisan.
Halaman sampul dalam terbuat dicetak pada kertas HVS. Pengetikan pada halaman judul
diketik simetris di bagian tengah (center). Logo STKIP Surya dengan ukuran panjang 6 cm
dan lebar 3,5 cm. Ukuran huruf tiap bagian tercantum pada format cover depan.
Pengecualian untuk nama program studi. Program Studi Pendidikan Fisika dan Matematika
dengan ukuran huruf 14, dan program studi Teknologi Informasi dan Komunikasi (TIK)
dengan ukuran huruf 12. Format dan contoh cover dalam dapat dilihat pada halaman
berikutnya.
%\newpage
%\includepdf[pages=-]{judulDepan}
\halamanJudul
\newpage
%\includepdf[pages=-]{judulDalam}
