% Chapter 1

\chapter{ini adalah judul yang sangat panjang Chapter Title Here} % Main chapter title


\section{Section}
\subsection{Sub Section}
\subsubsection{Sub sub section}


\ttitle

\begin{figure}[htbp]
\caption{ini adafdal'kj 'a;kldjkj'lkjk ';klj3 23'2lk3j  'kl;j'klj ''kjk' kj }
\end{figure}

\begin{table}[htbp]
\caption{sekedar contoh tabel d'pflasdj'fpklj 9d87y908e89hr ;klmn v}
\begin{tabular}{lllll}
3333 & iljghlkjh & ;oh;ljkh&;lkjlk \\
3333 & iljghlkjh & ;oh;ljkh&;lkjlk \\
3333 & iljghlkjh & ;oh;ljkh&;lkjlk \\
3333 & iljghlkjh & ;oh;ljkh&;lkjlk \\
3333 & iljghlkjh & ;oh;ljkh&;lkjlk \\
3333 & iljghlkjh & ;oh;ljkh&;lkjlk \\
3333 & iljghlkjh & ;oh;ljkh&;lkjlk \\
\end{tabular}
\end{table}

Sistematikanya adalah sebagai
berikut:
\begin{enumerate}
\item Judul cover depan
\item Judul cover dalam
\item Pernyataan keaslian skripsi
\item Halaman pengesahan
\item Kata pengantar
\item Abstrak
\item Daftar Isi
\item Daftar Tabel (jika ada)
\item Daftar Gambar (jika ada)
\item Daftar Simbol (seperti simbol matematika dan statistika, jika diperlukan)
\item Daftar Lampiran
\item Bab I. Pendahuluan
\item Bab II. Kajian Pustaka
\item Bab III. Metode Penelitian
\item Bab IV. Analisis Temuan dan Pembahasan
\item Bab V. Kesimpulan dan Saran
\item Daftar Pustaka
\item Lampiran
\end{enumerate}
