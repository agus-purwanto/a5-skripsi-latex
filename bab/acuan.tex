\chapter{Acuan}

Banyak style yang dapat digunakan untuk mengacu hasil penelitian 
yang telah dilakukan terdahulu.  Namun untuk skripsi 
di \institusiTulis, style yang digunakan adalah APA
(American Psychological Association).
Pemilihan style ini adalah karena APA sering digunakan 
dalam bidang pendidikan \autocite{styleGuide}.
Rincian style ini dapat dipelajari dari berbagai literatur,
termasuk yang ada di website \cite{APAPurdue}.
Usahakan untuk tidak mengacu website karena 
keberadaan website yang tidak jelas waktunya dan tidak memerlukan
adanya jaminan kualitas.

Tahapan penggunaan biblatex dengan style APA adalah sebagai berikut:
\begin{enumerate}
\item Pada preamble (di atas begin\{document\}) tuliskan:
\begin{Verbatim}[numbers=left,xleftmargin=5mm]
\usepackage[style=apa]{biblatex} 
\DeclareLanguageMapping{bahasai}{american-apa}

\addbibresource{example.bib} 
% nama file untuk daftar pustaka

\usepackage[autostyle=true]{csquotes} 
% perlu untuk kutipan bebas bahasa
\end{Verbatim}
Sebagai catatan, sebelumnya (melalui skripsiSTKIP.cls), telah 
dicantumkan 
\begin{Verbatim}[numbers=left,xleftmargin=5mm]
\usepackage[bahasai]{babel}
\end{Verbatim}
yang berguna untuk pemisahan suku kata sesuai dengan kaidah bahasa Indonesia.
\item Ditempat dimana daftar acuan akan dimunculkan,
tulislah:
\begin{Verbatim}[numbers=left,xleftmargin=5mm]
\printbibliography
\end{Verbatim}
\item File example.bib di atas, bisa mengandung:
\begin{Verbatim}[numbers=left,xleftmargin=5mm]
@article{Reference1,
	Abstract = {We have developed an enhanced Littrow configuration extended cavity diode laser (ECDL) that can be tuned without changing the direction of the output beam. The output of a conventional Littrow ECDL is reflected from a plane mirror fixed parallel to the tuning diffraction grating. Using a free-space Michelson wavemeter to measure the laser wavelength, we can tune the laser over a range greater than 10 nm without any alteration of alignment.},
	Author = {C. J. Hawthorn and K. P. Weber and R. E. Scholten},
	Journal = {Review of Scientific Instruments},
	Month = {12},
	Number = {12},
	Numpages = {3},
	Pages = {4477--4479},
	Title = {Littrow Configuration Tunable External Cavity Diode Laser with Fixed Direction Output Beam},
	Volume = {72},
	Url = {http://link.aip.org/link/?RSI/72/4477/1},
	Year = {2001}}

@article{Reference3,
	Abstract = {Operating a laser diode in an extended cavity which provides frequency-selective feedback is a very effective method of reducing the laser's linewidth and improving its tunability. We have developed an extremely simple laser of this type, built from inexpensive commercial components with only a few minor modifications. A 780~nm laser built to this design has an output power of 80~mW, a linewidth of 350~kHz, and it has been continuously locked to a Doppler-free rubidium transition for several days.},
	Author = {A. S. Arnold and J. S. Wilson and M. G. Boshier and J. Smith},
	Journal = {Review of Scientific Instruments},
	Month = {3},
	Number = {3},
	Numpages = {4},
	Pages = {1236--1239},
	Title = {A Simple Extended-Cavity Diode Laser},
	Volume = {69},
	Url = {http://link.aip.org/link/?RSI/69/1236/1},
	Year = {1998}}

@article{Reference2,
	Abstract = {We present a review of the use of diode lasers in atomic physics with an extensive list of references. We discuss the relevant characteristics of diode lasers and explain how to purchase and use them. We also review the various techniques that have been used to control and narrow the spectral outputs of diode lasers. Finally we present a number of examples illustrating the use of diode lasers in atomic physics experiments. Review of Scientific Instruments is copyrighted by The American Institute of Physics.},
	Author = {Carl E. Wieman and Leo Hollberg},
	Journal = {Review of Scientific Instruments},
	Keywords = {Diode Laser},
	Month = {1},
	Number = {1},
	Numpages = {20},
	Pages = {1--20},
	Title = {Using Diode Lasers for Atomic Physics},
	Volume = {62},
	Url = {http://link.aip.org/link/?RSI/62/1/1},
	Year = {1991}}

@online{styleGuide,
author = {University Library, American University},
title = {Which style should I use ?},
Url = {http://subjectguides.library.american.edu/c.php?g=175008&p=1154150},
urldate = {2016-01-03}
}

@online{APAPurdue,
title = {Purdue Online Writing Lab},
Url = {https://owl.english.purdue.edu/owl/resource/560/1/},
urldate = {2016-01-03}
}
\end{Verbatim}

\item Cara menggunakan / memanggil acuan adalah dengan menggunakan 
autocite, misalnya:
\begin{Verbatim}[numbers=left,xleftmargin=5mm]
Rincian style ini dapat dipelajari dari berbagai literatur,
termasuk yang ada di website \autocite{APAPurdue}.
\end{Verbatim}
\end{enumerate}

\section{Cara mengacu dengan biber/biblatex}
Yang paling sering diperlukan, paling tidak ada 2, yaitu:
\begin{enumerate}
\item Dengan menuliskan
\begin{Verbatim}[numbers=left,xleftmargin=5mm]
bla bla\autocite{Reference1,Reference2,Reference3}
\end{Verbatim}
akan diperoleh teks 
bla bla\autocite{Reference1,Reference2,Reference3}.
\item Dengan menuliskan
\begin{Verbatim}[numbers=left,xleftmargin=5mm]
seperti dibahas oleh 
\textcite{Reference1,Reference2,Reference3}.
\end{Verbatim}
akan diperoleh teks
seperti dibahas oleh \textcite{Reference1,Reference2,Reference3}.
\end{enumerate}

\section{Cara kompile acuan}
Ada 4 jurus, yaitu:
\begin{enumerate}
\item \texttt{xelatex skripsi}
\item \texttt{biber skripsi}
\item \texttt{xelatex skripsi}
\item \texttt{xelatex skripsi}
\end{enumerate}
Beberapa penjelasannya adalah
\begin{enumerate}
\item Tulisan \texttt{skripsi} di atas adalah 
nama file berakhiran \texttt{tex} yang akan 
di \textit{typeset} dengan \LaTeX.
\item Walau file bibliography yang berakhiran 
\texttt{bib} bernama file \texttt{example.bib},
pernyataan \texttt{biber} tetap mengacu ke nama file
yang memanggil \texttt{example.bib} (dalam hal ini file 
\texttt{skripsi} 
di atas).  
\item Setelah pernyataan \texttt{biber skripsi} di atas,
harus dilakukan dua kali \texttt{xelatex skripsi},
karena 
\begin{enumerate}
\item tahap pertama adalah penulisan ke file berakhiran
\texttt{bbl}
\item tahap kedua adalah penulisan ke file pdf.
\end{enumerate}
\end{enumerate}
