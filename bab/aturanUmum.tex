\chapter{Aturan Umum}
\section{Kertas}
Spesifikasi kertas yang digunakan:
\begin{enumerate} 
\item Jenis : HVS
\item Warna : Putih polos
\item Berat : 80 gram
\item  Ukuran : A5 (148 mm x 210 mm)
\end{enumerate}

\section{Teknik Pengetikan}
Ketentuan pengetikan adalah sebagai berikut:\autocite{Reference1,Reference2,Reference3}
\begin{enumerate}
\item Pencetakan dilakukan bolak balik.
\item Posisi penempatan teks pada tepi kertas:
\begin{enumerate}
\item Batas kiri : 4 cm tepi kertas
\item Batas kanan : 3 cm dari tepi kertas
\item Batas atas : 4 cm dari tepi kertas
\item Batas bawah : 3 cm dari tepi kertas
\end{enumerate}
\item Huruf menggunakan jenis huruf Times New Roman ukuran 12 dan diketik rata kiri kanan
(justify).
\item Pengetikan dilakukan dengan spasi 1,5
\end{enumerate}
\section{Penomoran Halaman}
\begin{enumerate}
\item Halaman Bagian awal:
Bagian awal skripsi diberi nomor halaman dengan menggunakan angka Romawi kecil (i,
ii, iii, dan seterusnya) ditempatkan pada posisi tengah bawah halaman yang dimulai dari judul
dalam (sesudah sampul) sampai dengan halaman Riwayat Hidup. Halaman judul dan halaman
persetujuan tidak diberi nomor, tetapi diperhitungkan sebagai halaman i dan ii yang tidak perlu
diketik.
\item Halaman Utama:
Penomoran mulai dari halaman Pendahuluan sampai dengan Kesimpulan dan Saran
menggunakan angka 1, 2, 3 dst. Setiap judul bab nomor diletakkan pada bagian tengah bawah dan
halaman berikutnya diketik di sudut kanan atas dengan jarak tiga spasi.
\item Halaman Bagian Akhir:
Penomoran pada bagian akhir karya ilmiah mulai dari Daftar Pustaka sampai dengan
Lampiran menggunakan angka. Penulisan diketik pada marjin bawah persis di tengah-tengah
dengan jarak tiga spasi dari marjin bawah teks pada setiap judul dan halaman selanjutnya diketik
di kanan atas dengan jarak tiga spasi dari pinggir atas (baris pertama teks) lurus dengan marjin
kanan teks.
\end{enumerate}
