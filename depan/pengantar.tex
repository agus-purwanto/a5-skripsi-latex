
%\begin{apresiasi}
\chapter*{Kata Pengantar}
\addchaptertocentry{Kata Pengantar} % Add the declaration to the table of contents



Pada kata pengantar dapat dikemukakan ucapan terima kasih dan apresiasi kepada pihak-pihak
yang telah membantu dalam menyelesaikan skripsi. 
%Teknik penulisan halaman kata pengantar secara umum adalah sebagai berikut.
%a.
%Semua huruf ditulis dengan jenis Times New Roman ukuran 12, spasi 1,5.
%b. b. Judul kata pengantar ditulis dengan tipe Times New Roman ukuran 12, dicetak tebal (bold)
%dan huruf kapital.
%c. c. Jarak antara judul dan isi kata pengantar adalah 4 spasi.
%Urutan pihak-pihak yang diberi ucapan terima kasih dimulai dari yang global menuju ke khusus, misalnya pihak STKIP, pihak luar
%(seperti sekolah tempat penelitian), keluarga atau teman.

Buku sederhana ini berfungsi sebagai panduan 
penulisan skripsi dengan menggunakan 
\LaTeX~berdasarkan \texttt{skripsiSTKIP.cls}.  
Ada banyak contoh praktis yang tersedia dalam pembahasan
di buku ini.
Namun untuk lengkapnya, pembaca dapat mencari 
file sumber nya yang berakhiran \texttt{tex} dan 
\texttt{bib}.  Beberapa hal menarik
yang diperoleh dari \LaTeX~dibandingkan dengan dari 
word processor seperti microsoft word adalah
\begin{enumerate}
\item Ada beberapa otomatisasi yang bisa dilakukan seperti
jumlah referensi, tahun awal-akhir referensi, banyaknya
tabel, banyaknya gambar, acuan dan \textit{cross reference}
yang dapat di klik.
\item \textit{Typeset} yang rapih, profesional dan seragam
untuk semua skripsi. Penggunaan \LaTeX~untuk 
publikasi ilmiah internasional sudah banyak dilakukan
di luar negeri.
\item Penggunaan \LaTeX~mengajarkan manusia untuk 
berkomunikasi dengan mesin (dalam hal ini komputer)
sehingga mesin bisa melakukan apa yang manusia tersebut
kehendaki.  Kemampuan komunikasi semacam ini semakin 
diperlukan di era revolusi industri 4.0 ini.
\end{enumerate}

\vfill
\begin{tabular}{lc}
\begin{minipage}{0.5\textwidth}
$\;$
\end{minipage}&
\begin{minipage}{0.4\textwidth}
\centering
Tangerang, \tanggalTulis ~\bulanTulis ~\tahunTulis \\ [12mm]
\penulisTulis\\
\nimTulis
\end{minipage}
\end{tabular}


%\end{apresiasi}

%\cleardoublepage

